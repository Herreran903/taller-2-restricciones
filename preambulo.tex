% --- Codificación y tipografías ---
\usepackage[T1]{fontenc}
\usepackage[utf8]{inputenc}
\usepackage[spanish,es-tabla]{babel}
\usepackage{lmodern}
\usepackage{microtype}

% --- Tamaño de fuente global ---
% (Usa el 11pt del \documentclass; si lo forzabas a 9pt, desactívalo)
% \AtBeginDocument{\fontsize{9}{10.5}\selectfont} % <- COMENTAR/ELIMINAR

% --- Márgenes más holgados ---
\usepackage{geometry}
\geometry{
  a4paper,
  top=2.2cm, bottom=2.5cm,
  left=2.2cm, right=2.2cm,
  headheight=13pt, headsep=12pt, footskip=20pt
}

% --- Gráficos ---
\usepackage{graphicx}
\graphicspath{{figuras/}{figs/}}

% --- Matemáticas y tablas ---
\usepackage{amsmath,amssymb}
\usepackage{booktabs}
\usepackage{siunitx}
\sisetup{
  detect-all,
  output-decimal-marker = {.},
  group-separator = {\,},
  group-minimum-digits = 4
}

% --- Flotantes (tablas/figuras) ---
\usepackage{float} % para [H]
\usepackage[font=small,labelfont=bf,skip=8pt]{caption}
\usepackage[section]{placeins}

% Espacios más cómodos para floats
\setlength{\textfloatsep}{14pt plus 4pt minus 3pt}
\setlength{\floatsep}{12pt plus 4pt minus 3pt}
\setlength{\intextsep}{12pt plus 4pt minus 3pt}

% Entorno opcional para compactar (si alguna sección lo necesita)
\newenvironment{compactfloats}{%
  \begingroup
  \setlength{\textfloatsep}{10pt plus 3pt minus 2pt}%
  \setlength{\floatsep}{10pt plus 3pt minus 2pt}%
  \setlength{\intextsep}{10pt plus 3pt minus 2pt}%
  \captionsetup{skip=6pt}%
  \renewcommand{\arraystretch}{1.0}%
  \setlength{\tabcolsep}{5pt}%
}{%
  \endgroup
}

% --- Bibliografía ---
\usepackage[numbers]{natbib}

% --- Enlaces y referencias cruzadas ---
\usepackage{hyperref}
\hypersetup{
  colorlinks=true,
  linkcolor=black,
  citecolor=black,
  urlcolor=black,
  pdfauthor={Equipo Taller 2},
  pdftitle={Taller 2 — Extensiones del Job Shop Scheduling}
}
\usepackage[nameinlink]{cleveref}

% Nombres en español para cleveref
\crefname{figure}{figura}{figuras}
\Crefname{figure}{Figura}{Figuras}
\crefname{table}{tabla}{tablas}
\Crefname{table}{Tabla}{Tablas}
\crefname{section}{sección}{secciones}
\Crefname{section}{Sección}{Secciones}

% --- Texto más “aireado” ---
\linespread{1.02}            % interlínea un poco mayor
\setlength{\parskip}{0.4em}  % espacio entre párrafos
\setlength{\parindent}{14pt} % sangría de párrafo

% Menos compresión alrededor de ecuaciones
\makeatletter
\g@addto@macro\normalsize{%
  \setlength\abovedisplayskip{10pt}%
  \setlength\belowdisplayskip{10pt}%
  \setlength\abovedisplayshortskip{8pt}%
  \setlength\belowdisplayshortskip{8pt}%
}
\makeatother

% --- Listas y títulos menos compactos ---
\usepackage{enumitem}
\setlist{leftmargin=*, itemsep=4pt, topsep=6pt, parsep=2pt, partopsep=0pt}

\usepackage[small,bf]{titlesec}
\titlespacing*{\section}{0pt}{12pt plus 4pt minus 2pt}{8pt}
\titlespacing*{\subsection}{0pt}{10pt plus 3pt minus 2pt}{6pt}
\titlespacing*{\subsubsection}{0pt}{8pt  plus 2pt minus 1pt}{5pt}
