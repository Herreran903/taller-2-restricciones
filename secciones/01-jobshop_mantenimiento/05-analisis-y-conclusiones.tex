% !TEX root = ../../main.tex

\subsection{Análisis y conclusiones}\label{sec:01-jobshop_mantenimiento-analisis-y-conclusiones}

En el solver Gecode, la efectividad de las restricciones redundantes depende de la estrategia de búsqueda aplicada. Con la heurística \texttt{first\_fail} (\texttt{ff\_min}), las redundantes resultaron mayormente perjudiciales. Incrementaron el tiempo de ejecución, el número de nodos explorados y los fallos, lo que evidencia que el costo de propagarlas no se compensa con una poda más efectiva del árbol de búsqueda.

Por el contrario, con la heurística informada \texttt{dom\_w\_deg} (\texttt{wdeg\_split}), se observó un beneficio claro, donde el número de nodos y fallos disminuyó y los tiempos mejoraron en la mayoría de las instancias, particularmente en las más complejas, aunque con algunas excepciones como \texttt{test\_10}. Esta estrategia, al ponderar el grado de conflicto de las variables, aprovecha mejor las restricciones redundantes para priorizar ramas más prometedoras. En cambio, con \texttt{input\_order} (\texttt{inorder\_min}), el impacto fue neutro o negativo. En varias ejecuciones el solver alcanzó el límite de tiempo sin mejoras visibles.

El solver Chuffed mostró un comportamiento diferente y más robusto frente a las restricciones redundantes. Con \texttt{ff\_min}, las redundantes ayudaron consistentemente a reducir el número de nodos y fallos, aunque el impacto en el tiempo fue irregular. Con \texttt{wdeg\_split}, su efecto volvió a ser positivo, logrando mejoras en las instancias más exigentes, con podas más tempranas del espacio de búsqueda. Finalmente, con \texttt{inorder\_min}, los resultados fueron mixtos, en donde algunas instancias presentaron leves mejoras y otras un aumento marginal en el tiempo, evidenciando que incluso en este caso Chuffed tolera mejor el costo de las redundantes que Gecode.

En síntesis, las restricciones redundantes R4 y R5 no son universalmente ventajosas: su impacto depende tanto de la heurística como del solver. Las estrategias informadas como \texttt{wdeg\_split} aprovechan mejor la información adicional para intensificar la poda, mientras que heurísticas más simples tienden a sufrir un costo de propagación que no se traduce en eficiencia. Chuffed se destacó por su capacidad de beneficiarse —o al menos no verse penalizado— por las redundantes en un rango más amplio de configuraciones, especialmente en las instancias de mayor complejidad.

De manera global, la combinación \texttt{dom\_w\_deg} + Chuffed resultó la más eficiente. Esta heurística, que prioriza variables con dominios pequeños y alta participación en conflictos, logra una poda temprana del árbol de búsqueda y una reducción sustancial en el número de nodos y fallos. Precisamente bajo esta configuración, las restricciones redundantes aportaron su mayor beneficio.

Las estrategias más simples (\texttt{ff\_min}, \texttt{inorder\_min}) mostraron un comportamiento mucho más variable y, en el caso de Gecode, incluso contraproducente, pues el esfuerzo de mantener las redundantes superó los beneficios en la poda.

En la comparación entre solvers, Chuffed fue sistemáticamente más rápido y eficiente que Gecode en la mayoría de los escenarios, particularmente con \texttt{wdeg\_split}. Su arquitectura basada en \textit{lazy clause generation} permite una propagación más inteligente y un mejor aprovechamiento de las redundantes, alcanzando soluciones óptimas en menos tiempo y con menor exploración.

Por lo tanto, la combinación Chuffed + \texttt{wdeg\_split} se perfila como la más recomendable para este modelo, pues aprovecha de forma conjunta las ventajas de una heurística informada y la información adicional de las restricciones redundantes (R4 y R5).

Respecto a la restricción de simetría (R6), se analizaron ejecuciones con y sin ella, manteniendo activas las redundantes. Dado que su efecto se limita a trabajos idénticos, el estudio se centró en las instancias \texttt{test\_01} y \texttt{test\_02}. En el contexto de optimización, R6 demostró ser claramente beneficiosa, sobre todo en \texttt{test\_02}. En Gecode, redujo nodos y fallos de forma sistemática, con mejoras notables en tiempo. En Chuffed, el impacto fue aún más marcado: con \texttt{ff\_min} el tiempo bajó de 35 ms a 8 ms, y con \texttt{wdeg\_split}, de 10 ms a 4 ms, acompañadas de significativas reducciones en nodos y fallos.

Para cuantificar su efecto sobre el espacio de búsqueda, se realizaron pruebas en modo \texttt{satisfy} para enumerar todas las soluciones posibles, comparando configuraciones con y sin R6. Aunque la mayoría alcanzó el límite de 60 s antes de completar la enumeración, en todos los casos el número de soluciones halladas antes del \textit{timeout} fue ligeramente menor con la simetría activa. La reducción no fue drástica, pues la simetría solo elimina permutaciones entre trabajos idénticos, pero confirma que R6 funciona correctamente.

En conclusión, la restricción de simetría R6 constituye una adición util para instancias con trabajos idénticos: mejora la optimización, reduce la redundancia de soluciones y acelera la búsqueda. Aunque su impacto en la enumeración completa se ve limitado por el tiempo y el alcance parcial de la simetría.
