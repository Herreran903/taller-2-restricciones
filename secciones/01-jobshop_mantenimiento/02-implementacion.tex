% !TEX root = ../../main.tex

\subsection{Implementación}\label{sec:01-jobshop_mantenimiento-implementacion}

\subsubsection*{Modelo}
El modelo captura de forma correcta la estructura del problema mediante los parámetros definidos en la sección anterior y las restricciones principales \(R1\)–\(R3\).  
Las variables \(s_{i,m}\) y \(\texttt{END}\) permiten representar explícitamente el instante de inicio y finalización de cada operación, de modo que cualquier configuración factible de estas variables corresponde a un cronograma real.  
La restricción \(R1\) asegura la correcta secuencia de operaciones dentro de cada trabajo, preservando el orden tecnológico sin permitir solapamientos entre tareas consecutivas del mismo job.  
La restricción \(R2\) implementa la capacidad unitaria de cada máquina, garantizando que solo una operación se ejecute a la vez en ella; esto se logra mediante la disyunción de no solape, lo que define implícitamente un orden válido entre operaciones que comparten recurso.  
Finalmente, \(R3\) extiende el modelo clásico incorporando las ventanas de mantenimiento: los intervalos definidos por \(\texttt{MAINT\_START}\), \(\texttt{MAINT\_END}\) y activados por \(\texttt{MAINT\_ACTIVE}\) se tratan como periodos ocupados dentro de la misma lógica de exclusión que entre operaciones.

\subsubsection*{Restricciones redundantes}
Las restricciones \(R4\)–\(R6\) son lógicamente implicadas por el modelo y sólo refuerzan la propagación.  
\(R4\) (cota por trabajo) obliga a \(\texttt{END}\) a ser al menos la suma de duraciones de cada trabajo, descartando de inmediato valores imposibles del objetivo.  
\(R5\) (carga por máquina) exige que \(\texttt{END}\) no sea menor que la carga total procesada por cada máquina.  
\(R6\) acota todas las variables al horizonte seguro \(H\); en la implementación esta cota se aplica de forma implícita mediante los dominios \([0,H]\) de \(s_{i,m}\) y \(\texttt{END}\).

\subsubsection*{Restricciones de simetría}
Cuando existen trabajos con la misma secuencia de duraciones (\(\texttt{PROC\_TIME}[i,*]=\texttt{PROC\_TIME}[k,*]\)), el problema admite soluciones equivalentes por simple intercambio de etiquetas.  
La restricción \(R7\) impone un orden léxico sobre los vectores de inicios \((s_{i,1},\dots,s_{i,|M|})\) para esos pares \(i<k\), de modo que se conserve un único representante por clase de permutación.