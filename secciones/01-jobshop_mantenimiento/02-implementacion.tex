% !TEX root = ../../main.tex

\subsection{Implementación}\label{sec:01-jobshop_mantenimiento-implementacion}

\subsubsection*{Modelo}
El modelo captura de forma correcta la estructura del problema mediante los parámetros definidos en la sección anterior y las restricciones principales \(R1\)–\(R3\).
Las variables \(s_{i,m}\) y \(\texttt{END}\) permiten representar explícitamente el instante de inicio y finalización de cada operación, de modo que cualquier configuración factible de estas variables corresponde a un cronograma real.
La restricción \(R1\) asegura la correcta secuencia de operaciones dentro de cada trabajo, preservando el orden tecnológico sin permitir solapamientos entre tareas consecutivas del mismo job.
La restricción \(R2\) implementa la capacidad unitaria de cada máquina, garantizando que solo una operación se ejecute a la vez en ella; esto se logra mediante la disyunción de no solape, lo que define implícitamente un orden válido entre operaciones que comparten recurso.
Finalmente, \(R3\) extiende el modelo clásico incorporando las ventanas de mantenimiento: los intervalos definidos por \(\texttt{MAINT\_START}\), \(\texttt{MAINT\_END}\) y activados por \(\texttt{MAINT\_ACTIVE}\) se tratan como periodos de \textit{no disponibilidad} de la máquina, impidiendo que cualquier operación se solape con ellos.

\subsubsection*{Restricciones redundantes}
Las restricciones \(R4\)–\(R6\) son \textbf{lógicamente implicadas} por las restricciones principales del modelo, es decir, no añaden información nueva sobre el conjunto de soluciones factibles. Su propósito es \textbf{reforzar la propagación} de cotas durante la búsqueda, ayudando al solver a podar ramas del árbol de exploración más temprano.
\(R4\) (cota por trabajo) obliga a \(\texttt{END}\) a ser al menos la suma de duraciones de cada trabajo individual, descartando de inmediato valores imposibles del objetivo.
\(R5\) (carga por máquina) exige que \(\texttt{END}\) no sea menor que la carga total de trabajo acumulada en cada máquina.
\(R6\) acota todas las variables al horizonte seguro \(H\); en la implementación esta cota se aplica de forma implícita mediante los dominios \([0,H]\) definidos para las variables \(s_{i,m}\) y \(\texttt{END}\).

\subsubsection*{Restricciones de simetría}
Cuando existen trabajos con la misma secuencia de duraciones (\(\texttt{PROC\_TIME}[i,*]=\texttt{PROC\_TIME}[k,*]\)), el problema admite soluciones equivalentes donde solo se intercambian las etiquetas de estos trabajos idénticos.
La restricción \(R7\) \textbf{rompe esta simetría} imponiendo un orden léxico sobre los vectores de tiempos de inicio \((s_{i,1},\dots,s_{i,|M|})\) para cada par de trabajos idénticos \(i<k\). De este modo, se conserva un único representante canónico por cada clase de permutación equivalente. Dado que los trabajos idénticos son intercambiables sin afectar la optimalidad del makespan, esta restricción \textbf{elimina soluciones estructuralmente idénticas sin perder la solución óptima}, por lo que no es necesario un proceso posterior para recuperar soluciones.