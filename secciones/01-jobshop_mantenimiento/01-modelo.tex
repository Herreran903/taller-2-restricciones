% !TEX root = ../../main.tex

\subsection{Modelo}\label{sec:01-jobshop_mantenimiento-modelo}

\subsubsection*{Parámetros}
\begin{description}
  \item[\textbf{P1 — \texttt{JOBS}:}] Cantidad de trabajos.
  \item[\textbf{P2 — \texttt{TASKS}:}] Cantidad de máquinas.
  \item[\textbf{P3 — \texttt{PROC\_TIME}:}] Matriz de duraciones \(p_{i,m}\) de tamaño \(\texttt{JOBS}\times\texttt{TASKS}\): \(\texttt{PROC\_TIME}[i,m]=p_{i,m}\).
  \item[\textbf{P4 — \texttt{MAX\_MAINT\_WINDOWS}:}] Tope global de ventanas de mantenimiento por máquina.
  \item[\textbf{P5 — \texttt{MAINT\_START}, \texttt{MAINT\_END}:}] Inicios \(a_{m,k}\) y fines \(b_{m,k}\) de cada ventana \(k\) en máquina \(m\).
  \item[\textbf{P6 — \texttt{MAINT\_ACTIVE}:}] Indicadores booleanos \(\texttt{MAINT\_ACTIVE}[m,k]\) que activan la ventana \([a_{m,k}, b_{m,k})\).
\end{description}

\subsubsection*{Constantes derivadas}
\begin{description}
  \item[\textbf{D1 — \(H\):}] Horizonte superior \emph{seguro}, derivado como
  \[
  H \;=\; \sum_{i=1}^{\texttt{JOBS}}\sum_{m=1}^{\texttt{TASKS}} p_{i,m}
  \;+\; \sum_{m=1}^{\texttt{TASKS}}\sum_{k=1}^{\texttt{MAX\_MAINT\_WINDOWS}}
  \bigl(b_{m,k}-a_{m,k}\bigr)\,\mathbf{1}\!\left[\texttt{MAINT\_ACTIVE}[m,k]\right].
  \]
  \item[\textbf{D2 — \(J\):}] Conjunto de trabajos, \(J=\{1,\dots,\texttt{JOBS}\}\).
  \item[\textbf{D3 — \(M\):}] Conjunto de máquinas, \(M=\{1,\dots,\texttt{TASKS}\}\).
\end{description}

\subsubsection*{Variables}
\begin{description}
  \item[\textbf{V1 — \(s_{i,m}\):}] Inicio de la operación del trabajo \(i\) en la máquina \(m\), con \(s_{i,m}\in[0,H]\).
  \item[\textbf{V2 — \(\texttt{END}\):}] \emph{Makespan} del programa, \(\texttt{END}\in[0,H]\).
\end{description}

\subsubsection*{Restricciones principales}
\begin{description}
  \item[\textbf{R1 — Precedencias dentro del trabajo:}] Las operaciones de cada trabajo siguen su orden dado.  
  \[
  \forall i\in J,\ \forall m\in\{1,\dots,|M|-1\}:\quad
  s_{i,m}+p_{i,m}\ \le\ s_{i,m+1},
  \qquad
  s_{i,|M|}+p_{i,|M|}\ \le\ \texttt{END}.
  \]

  \item[\textbf{R2 — No solape por máquina:}] En cada máquina, las operaciones se procesan de a una (restricción disyuntiva).  
  \[
  \forall m\in M,\ \forall i,k\in J,\ i<k:\quad
  \big(s_{i,m}+p_{i,m}\le s_{k,m}\big)\ \lor\ \big(s_{k,m}+p_{k,m}\le s_{i,m}\big).
  \]

  \item[\textbf{R3 — Bloqueos por mantenimiento:}]
  Ninguna operación se ejecuta durante una ventana activa de mantenimiento.  
  \begin{align}
  & \forall m\in M,\ \forall k\in\{1,\dots,\texttt{MAX\_MAINT\_WINDOWS}\}\ \text{con }\texttt{MAINT\_ACTIVE}[m,k]=\texttt{true},\ \forall i\in J: \nonumber\\
  & \quad (s_{i,m}+p_{i,m}\le a_{m,k})\ \lor\ (b_{m,k}\le s_{i,m})\;, \nonumber
  \end{align}
  donde \(a_{m,k}=\texttt{MAINT\_START}[m,k]\) y \(b_{m,k}=\texttt{MAINT\_END}[m,k]\), con \(0\le a_{m,k}<b_{m,k}\le H\).  
\end{description}

\subsubsection*{Restricciones redundantes}
\begin{description}
  \item[\textbf{R4 — Cota por trabajo:}] El \emph{makespan} no puede ser menor que la suma de duraciones de cada trabajo.  
  \[
  \forall i\in J:\quad \texttt{END}\ \ge\ \sum_{m\in M} p_{i,m}.
  \]

  \item[\textbf{R5 — Carga por máquina:}] El \emph{makespan} acota inferiormente la carga total de cada máquina.  
  \[
  \forall m\in M:\quad \texttt{END}\ \ge\ \sum_{i\in J} p_{i,m}.
  \]

  \item[\textbf{R6 — Cota por horizonte:}] Las fechas de inicio y el \emph{makespan} se restringen al horizonte \(H\).  
  \[
  \forall i\in J,\ \forall m\in M:\ 0\le s_{i,m}\le H, 
  \qquad 0\le \texttt{END}\le H.
  \]
\end{description}

\subsubsection*{Restricciones de simetría}
\begin{description}
  \item[\textbf{R7 — Trabajos idénticos:}] Para evitar permutaciones equivalentes, si \(p_{i,*}=p_{k,*}\) y \(i<k\) se impone orden léxico en los inicios:
  \[
  (s_{i,1},\dots,s_{i,|M|}) \le_{\text{lex}} (s_{k,1},\dots,s_{k,|M|}).
  \]
\end{description}